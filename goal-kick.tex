\section{The Goal Kick}\label{sec:goal-kick}

A goal kick is a method of restarting play.

A goal may be scored directly from a goal kick, but only against the opposing team.

A goal kick is awarded when:

\begin{itemize}
\item the whole of the ball, having last touched a robot of the attacking team, passes over the goal line, either on the ground or in the air, and a goal is not scored in accordance with \autoref{sec:method-of-scoring}
\end{itemize}

\subsection{Procedure}
\begin{itemize}
\item the ball is kicked from a point 500\,mm from the goal line and 100\,mm from the touch line closest to where the ball passed over the goal boundary
\item opponents remain 500\,mm from the ball until the ball is in play
\item the kicker does not play the ball a second time until it has touched another robot
\item the ball is in play when it is kicked and moves
\item \added{if the goal kick is kicked directly into the opposing goal, but the ball traveled above 150\,mm from the ground, the goal kick is retaken}
\end{itemize}

\subsection{Infringements/Sanctions}
Any infringement as listed in \autoref{sec:ball-in-and-out-of-play} is handled accordingly\added{.}

For any other infringement of this Law:
\begin{itemize}
\item the kick is retaken
\end{itemize}
