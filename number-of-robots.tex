\section{The Number of Robots}\label{sec:number-of-robots}

\subsection{Robots}
A match is played by two teams, each consisting of not more than six robots, one of which may be the goalkeeper.
Each robot must be clearly numbered so that the referee can identify them during the match.
The goalkeeper must be designated before the match starts.
A match may not start unless both teams have at least one robot.

\subsubsection{Interchange}\label{subsubsec:number-of-robots-interchange}
Robots may be interchanged.
There is no limit on the number of interchanges.

\subsubsection{Interchange Procedure}
To interchange a robot, the following conditions must be observed:
\begin{itemize}
\item interchange can only be made during a stoppage in play,
\item the referee is informed before the proposed interchange is made,
\item the interchange robot enters the field of play after the robot being replaced has been removed,
\item the interchange robot enters the field of play at the halfway line.
\end{itemize}

\subsubsection{Changing the Goalkeeper}
Any of the other robots may change places with the goalkeeper, provided that:
\begin{itemize}
\item the referee is informed before the change is made of which robot will be the new goalie
\item the change is made during a stoppage in the match
\item the referee indicates the number of the new goalie, which is sent by communication link to the teams
\end{itemize}

\subsection*{Decisions of the Small Size League Technical Committee}
\begin{enumerate}
\item
Each team must have a single designated robot handler to perform interchange and robot placing when required.
No other team members can encroach upon the area immediately surrounding the field.
Movement of robots by the handler is not allowed.
\end{enumerate}