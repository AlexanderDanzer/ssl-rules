\section{The Start and Restart of Play}\label{sec:start-and-restart-of-play}

\subsection{Preliminaries}
If both teams have a common preferred frequency for wireless communications, the local organising committee will allocate that frequency for the first half of the match.
If both teams have a common preferred color, the local organising committee will allocate the color for the first half of the match.

A coin is tossed and the team which wins the toss decides which goal it will attack in the first half of the match.

The other team takes the kick-off to start the match.

The team that wins the toss takes the kick-off to start the second half of the match.

In the second half of the match the teams change ends and attack the opposite goals.
Teams may agree not to change ends and attack the opposite goals with the consent of the referee.

If both teams have a common preferred frequency for wireless communications, the teams should swap the allocation of that frequency for the second half of the match.
Teams may agree not to change the allocation of the preferred frequency with the consent of the referee.

If both teams have a common preferred marker color, the teams should swap marker colors for the second half of the match.
Teams may agree not to change the marker colors with the consent of the referee.

\subsection{Kick-off}
A kick-off is a way of starting or restarting play:
\begin{itemize}
\item at the start of the match
\item after a goal has been scored
\item at the start of the second half of the match
\item at the start of each period of extra time, where applicable
\end{itemize}

A goal may be scored directly from the kick-off.

\subsection{Procedure}
\begin{itemize}
\item all robots are in their own half of the field
\item the opponents of the team taking the kick-off are at least 500\,mm from the ball until the ball is in play
\item the ball is stationary on the centre mark
\item the referee gives a signal
\item the ball is in play when is kicked and moves forward
\item the kicker does not touch the ball a second time until it has touched another robot
\end{itemize}

After a team scores a goal, the kick-off is taken by the other team.

\subsection{Infringements/Sanctions}
Any infringement as listed in \autoref{sec:ball-in-and-out-of-play} is handled accordingly.

For any other infringement of the kick-off procedure:
\begin{itemize}
\item the kick-off is retaken
\end{itemize}

\subsection{Placed Ball}
A placed ball is a way of restarting the match after a temporary stoppage which becomes necessary, while the ball is in play, for any reason not mentioned elsewhere in the Laws of the Game.

\subsection{Procedure}
The referee places the ball at the place where it was located when play was stopped.
By Law 9, all robots are required to remain 500\,mm from the ball while the ball is being placed.
Play restarts when the referee gives a signal.

\subsection{Infringements/Sanctions}
The ball is placed again:
\begin{itemize}
\item if a robot comes within 500\,mm of the ball before the referee gives the signal
\end{itemize}

\subsection{Special Circumstances}
A free kick awarded to the defending team inside its own defence area is taken from a legal free kick position nearest to where the infringement occurred.

A free kick awarded to the attacking team in its opponents' defence area is taken from a legal free kick position nearest to where the infringement occurred.

A placed ball to restart the match after play has been temporarily stopped inside the defence area takes place on the legal free kick position nearest to where the ball was located when play was stopped.
