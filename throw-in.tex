\section{The Throw-In}\label{sec:throw-in}

A throw-in is a method of restarting play.

A goal cannot be scored directly from a throw-in.

\begin{itemize}
\item if a throw-in is kicked directly into the opponents' goal, a goal kick is awarded\added{ to the defending team}
\item if a throw-in is kicked directly into the team's own goal, a corner kick is awarded to the opposing team
\end{itemize}

A throw-in is awarded:

\begin{itemize}
\item when the whole of the ball passes over the touch boundary, either on the ground or in the air
\item from the point 100\,mm perpendicular to the touch boundary where the ball crossed the touch boundary
\item to the opponents of the robot that last touched the ball
\end{itemize}

\subsection{Procedure}
\begin{itemize}
\item the referee places the ball at the designated position
\item all opponent robots are at least 500\,mm from the ball
\item the ball is in play when it is kicked and moves
\end{itemize}

\subsection{Infringements/Sanctions}
If, when a throw-in is taken, an opponent is closer to the ball than the required distance:

\begin{itemize}
\item the throw-in is retaken
\end{itemize}

Any infringement as listed in \autoref{sec:ball-in-and-out-of-play} is handled accordingly.

For any other infringement:

\begin{itemize}
\item the kick is retaken
\end{itemize}
